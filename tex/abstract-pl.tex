Zagadnienie rozumowania o systemach komputerowych jest niezmiennie ważne. Rozumowanie o hierarchicznych platformach sterowania z elementami zamiarowego wykonywania jest ponadprzeciętnie trudne. Znaczącym źródłem komplikacji jest użycie modelu zamiarowego wykonywania (ang. intent-based actuation). Popularny mechanizm śledzenia (ang. tracing) zapytań w systemach rozproszonych OpenTracing został stworzony, przede wszystkim, do śledzenia zapytań w systemach serwujących (ang. online serving systems). Jednakże OpenTracing nie obsługuje śledzenia aktywności systemu, w momencie gdy zachodzi efekt łączenia aktywności (ang. coalescing effect) z różnych drzew zapytań (ang. request traces). Takie sytuacje występują często w platformach sterowania systemami komputerowymi, m.in chmurą, oraz w systemach kompilacji (ang. build systems). Platformy sterowania chmurą same często posiadają w sobie elementy systemów kompilacji. Żadne obecnie rozwiązania nie obsługują specyfiki hierarchicznych platform sterowania z elementami zamiarówego wykonywania.  Celem pracy jest rozwiązanie tego problemu poprzez stworzenie nowatorskiej metody śledzenia zadań w systemach rozproszonych opartej o nowy model, będący rozszerzeniem modelu OpenTracing. Istniejące badania w tematach pokrewnych m.in. śledzenie proweniencja (ang. provenance tracking) dostarczają elementów które możemy wykorzystać do rozwiązania problemu. System debugowania dla hierarchicznych platform sterowania musi spełniać specyficzne wymagania. Na ich podstawie został opracowany nowy model śledzenia zdarzeń z elementami śledzenia proweniencja, wraz ze szkicem jego formalizacji na podstawie teorii grafów. Model ten radzi sobie z problemem łączenia wydarzeń z różnych drzew zapytań. Wraz z modelem powstała architektura spełniająca powyższe wymagania. Na podstawie modelu i architektury zostało stworzone narzędzie Tenmo, oparte o instrumentacje oraz rozproszony i asynchroniczny mechanizm zbierania i agregowania danych śledzenia z systemów rozproszonych. Stworzone narzędzie zostało zweryfikowane poprzez jego zastosowanie do reprezentatywnego zbioru oprogramowania odzwierciedlającego główne elementy składowe hierarhicznych platform sterowania chmurą. Opisana nowatorska metoda rozwiązuje problem śledzenia hierarhicznych platform sterowania z elementami zamiarowego oraz zajmuje wcześniej puste miejsce w przestrzeni rozwiązań debugowania systemów komputerowych.

\noindent \textbf{Słowa kluczone:} Metody śledzenia w rozproszonych systemach. Proweniencja obiektów w platformach sterowania systemów komputerowych. Systemy rozproszone. Chmura. PaaS. IaaS. Deklaratywne zarządzanie infrastrukturą. Systemy kompilacji.
