\newpage
\chapter{Architecture}\label{sec:arch}

The model proposed in \cref{sec:pedst-model} is not sufficient on its own to solve the problem stated in \cref{sec:problem}. The model has to be supported by a solution architecture which satisfies the requirements. In this section we describe such architecture, which is Provenance-Enhanced Distributed Systems Tracing architecture (see \cref{fig:arch}).

\begin{figure}[htpb]
    \centering
    %%TODO: Replace with Tikz nicer image
\digraph[scale=0.6]{arch8}{
    rankdir=TB;
    subgraph cluster_db {
        events [label="Events"];
        model [label="Provenance"];
        label="Storage";
    }
    client [label="Traced system"];
    pipeline [label="Ingestion pipeline"];
    client -> pipeline [label=" logs events"];
    pipeline -> events [label=" collects events"];
    events -> processor [label=" receives events"];
    processor -> processor [label=" transforms events to records"];
    processor -> model [label=" stores records"];
    model -> query [label=" queries"];
    query -> viewer [label=" shows results"];
}
    \caption{Data flow in the Provenance-Enhanced Distributed Systems Tracing architecture.}
    \label{fig:arch}
\end{figure}

The PEDST architecture consists of:
%
\begin{itemize}
	\item Global data model schema to capture the global view of what has been recorded among all systems participating in the data gathering process (see,~\cref{sec:global-tuples}).
	\item Logging data model schema for structured logging to be performed by each individual component participating in the data gathering process (see,~\cref{sec:local-model}).
	\item Storage system to store events (using logging schema) and records (using global data schema).
	\item Logging protocol for event ingestion.
	\item Ingestion pipeline which ingests events from event producers and stores it in the storage system.
	\item Events processor which uses events to build up global data based on it, and stores it in the storage system.
	\item Set of queries defined over global data.
	\item Visualization tool using of the queries' results of the global data.
\end{itemize}

The model and the architecture are still not a solution for the problem at hand. The data produced by a traced system has to follow a structured logging protocol and data schema described in \fullref{sec:logging-model}. Even if the protocol is followed, the PEDST model allows for a considerable freedom how data is logged, hence a set of recommendations how to perform logging from traced application is presented in \fullref{sec:patterns}.

To satisfy ``Supports granular fidelity implementation'', ``Supports incremental implementation'' and ``Low storage overhead'' requirements (see,~\cref{sec:reqs}), PEDST architecture assumes a white-box tracing approach. This means that tracing and logging instrumentation is actively engineered into the software whose actions are being inspected. This means that we need to provide a protocol to send events (see,~\cref{sec:logging-model}) from the traced system to a PEDST implementation deployment. Additionally, we have to provide a logging library, to simplify usage of the protocol. The white-box tracing focus puts this work in a ``disclosed provenance'' category~\cite{braun2006issues}.

As per ``Limited trust'' non-goal, the proposed system is designed to be used in the confines of a single stakeholder’s infrastructure. Hence, we are not aiming for a tamper-resistant system and are trusting events logged by and within a traced system.

\paragraph*{Ingestion pipeline.}

Events logged by a traced system are received by an ingestion pipeline (see,~\cref{sec:ingestion-pipeline} for details), which gathers events from all components of the system and stores them in the event storage. The main goal of the ingestion pipeline is a durable delivery of events from all components of the, usually distributed, system under tracing. Hence, it contributes to the ``Full coverage'' requirement. Depending on the requirements, and the scale of the traced system, the ingestion system might be as simple as a shared NFS directory for structured log files with JSON entries, a large Splunk ingestion deployment, or as complex as an ingestion pipeline, used by Google, for Dapper traces. An asynchronous ingestion pipeline satisfies the ``Asynchronous data intake'' requirement and contributes to the ``Event-based data production'' requirement. 

The distributed ingestion pipeline stores the events gathered from the traced system in an event store. The event store is essentially a queue for a processor to act upon. Presence of a distributed ingestion pipeline satisfies the ``cross-host tracking`` requirement by ingesting events from all hosts participating in the traced system.

\paragraph*{Events processor.}

The processor takes out an event from the events store, processes it, and deletes it. In case of a failure to process an event (e.g. if the event depends on something which is not yet recorded), it should be -- after a delay -- stored back in the queue. If it is impossible to process an event after a specified time, the event is discarded.

Processing of events happens according to the procedure defined in~\fullref{sec:local-model}. Whenever processing of an event needs new records to be stored, they are added to the records store.

Events processor may be responsible to transform incoming provenance information into a more convenient or efficient representation and store it in the storage system. The representation should be suitable for efficient graph queries.

%% Should I make RDF representation of the graph part of the architecture? Or should it be an implementation detail?
%% I am keeping it as an implementation detail, since there are no *direct* requirements forcing us to use the RDF representation.

\paragraph*{Storage.}

Records store stores tuples, which represent an observed universe, defined in~\fullref{sec:global-tuples}. The storage should also be used to store a query-optimized representation of the same data.

\paragraph*{Queries.}

Queries over the record store allow to construct any of the graph representations of the observed universe (see ~\cref{sec:global-graphs}) and their extensions (see ~\cref{sec:model-extensions}).

\bigskip

In \fullref{sec:impl} we present a proof of concept implementation of the PEDST model and architecture.